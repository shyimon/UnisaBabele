Il protocollo IP è il protocollo principale della rete Internet.

\section{Internetworking}
    Gli strati fisici e di collegamento operano localmente. Questi hanno la possibilità di consegnare un pacchetto dati al prossimo nodo, sulla stessa rete.
    
    In una interrete un pacchetto deve passare da router e switches che lo instradano correttamente, e che devono chiaramente sapere su quale nodo inviarlo.
    
    Per risolvere questo problema non sono sufficienti gli strati fisico e di collegamento, e abbiamo \textbf{bisogno di uno strato di rete}, il quale ha esattamente il compito di far arrivare i pacchetti dati dal nodo mittente al nodo destinatario, eventualmetne passando per i corretti switch e router. Questo consta nel controllare le tavole di routing per instradare correttamente il pacchetto ed eventualmente frammentarlo se questo è troppo grande. Una volta arrivato a destinazione, lo strato di rete ha il compito di controllare se il pacchetto sia effettivamente arrivato alla destinazione corretta, e se è un frammento, di aspettare gli altri per ricomporre il messaggio originale.
    
    \subsection{Internet e reti a datagram}
        La rete internet utilizza la commutazione di pacchetto con approccio a datagram per operare la commutazione nello strato di rete. Vengono usati gli indirizzi IP come indirizzi logici.
        
    \subsection{Internete e reti senza connessione}
        La consegna di pacchetti a un destinatario possono essere fatti utilizzando servizi con o senza connessione. In un servizio \textbf{con connessione}, va prima instaurata appunto una connessione. In questo caso i pacchetti hanno una relazione: ogni pacchetto è connesso logicamente alla connessione a cui appartiene, al pacchetto che lo precede e a quello successivo. Quando tutti i pacchetti sono stati consegnati la connessione può essere eliminata.
        
        In una comunicazione con connessione, la decisione del cammino che i pacchetti dovranno seguire viene fatta solo una volta, ossia quando la connessione stessa viene instaurata. I pacchetti viaggiano in ordine su un cammino prestabilito.
        
        In un \textbf{servizio senza connessione} non offre nessun collegamento fra i pacchetti: due pacchetti che hanno in comune mittente e destinatario e che appartengono allo stesso messaggio potrebbero seguire due strade diverse. Questo è l'approccio adottato dalla rete Internet, in quanto essa è composta da sistemi troppo eterogenei e sarebbe difficile instaurare connessioni.
        
\section{IPv4}
    Il protocollo IPv4 implementa il servizio di consegna dei pacchetti senza connessione. È uno dei protocolli principali della suite TCP/IP.
    
    Esso offre un servizio di consegna detto \textbf{best effort}. La consegna avviene senza connessione ed è inaffidabile. Non viene offerto nessun controllo riguardante gli errori ed il flusso. Si noti anche che non necessariamente i pacchetti arriveranno nell'ordine in cui sono stati spediti.
    
    Qualora ci sia necessità di un servizio più affidabile o con maggiori funzionalità ci si deve affidare a un protocollo di livello superiore, come quello TCP.
    
    \subsection{Datagram}
        I pacchetti del protocollo IPv4 vengono chiamati \textbf{datagram}. Essi hanno lunghezza variabile e sono divisi in due parti: un'intestazione (che varia da 20 a 60 byte) e un carico dati. L'intestazione è strutturata come segue:
        \begin{enumerate}
            \item \textbf{Versione.} La versione del protocollo IP. Fornisce contesto e giusta interpretazione per tutti i campi successivi. Se non è riconosciuta il datagram viene scartato.
            
            \item \textbf{LI (lunghezza intestazione).} Grande 4 bit, descrive la lunghezza dell'intestazione in gruppi di parole da 4 byte: quindi la lunghezza sarà un multiplo di 4. Può andare da 20 a 60 byte.
            
            \item \textbf{Servizi.} Questi bit rappresentano la priorità del pacchetto in situazioni di congestione e il servizio da offrire (affidabilità massima, costo minimo, etc.)
        \end{enumerate}
        
        Altri campi indicano la lunghezza totale del datagram, il suo Time To Live, eventuali flag e indirizzi di mittente e destinatario.
        
    \subsection{Frammentazione}
        La grandezza di un datagram deve essere inferiore alla MTU (Maximum Transfer Unit) e dipende dallo strato fisico, in quanto legata a caratteristiche hardware. Per rendere il protocollo IPv4 indipendente dall'hardware, si è deciso di stabilire una grandezza massima dei datagram di 65535 byte e di frammentarlo in datagram più piccoli qualora questa condizione venisse violata per eccesso.
        
        Questa operazione di frammentazione può avvenire anche nel nodo sorgente prima dell'invio. Un datagram è frammentato in quelli che sono a loro volta altri datagram a tutti gli effetti, i quali possono essere ulteriormente frammentati se si trovano a passare da stazioni con MTU più piccoli.
        
        Il \textbf{riassemblaggio} avviene invece solo al nodo sorgente. Questo è infatti l'unico nodo su cui sappiamo per certo che passano tutti i frammenti, a meno di errori o perdite.
        
    \subsection{Campi dell'intestazione usati per la frammentazione}
        L'\textbf{identificativo}, insieme con l'indirizzo IP della sorgente servono a identificare univocamente un datagram. Questo viene mantenuto nei frammenti e quindi è utile per identificarli come appartenenti a un unico datagram.
        
        Il \textbf{flag} è composto da 3 bit: il primo è riservato, il secondo è chiamato di \textit{non frammentazione}, che indica se il datagram deve essere ulteriormente frammentato, e il terzo è chiamato \textit{bit di altri frammenti}; se questo bit vale 1 ci sono altri frammenti di questo datagram, ossia il presente non è l'ultimo frammento.
        
        L'\textbf{offset} indica il valore di offset rispetto al datagram originale in unità di 8 byte; se il frammento è un frammento di un frammento, questo valore fa sempre riferimento al datagram originale.
        
        Per ricomporre un datagram si segue la seguente strategia:
        \begin{enumerate}
            \item Si cerca il frammento con offset 0, che sarà il primo.
            
            \item Se il primo frammento ha come bit di altri frammenti 1, si divide la sua lunghezza per 8 e si cerca un frammento con quel valore come offset.
            
            \item Si continua ad andare avanti cercando frammenti in base all'offset finché non se ne trova uno con bit di altri frammenti uguale a 0.
        \end{enumerate}
        
    \subsection{Opzioni}
        L'intestazione di un datagram IPv4 contiene due sezioni, una fissa e una variabile. Quella fissa è lunga 20 byte ed è stata descritta, mentre quella variabile può arrivare a 40 byte e può opzionalmente contenere opzioni che tutti i software che trattano pacchetti IPv4 devono sapere come gestire.
        
        Queste opzioni vengono usate per debugging od operazioni di testing della rete. Includono per esempio la registrazione dei router attraversati, un instradamento guidato o suggerito dalla sorgente, o registrazioni di orario.