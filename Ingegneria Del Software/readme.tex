    \vspace{3mm}
    \section{Filosofia di questi appunti e perché dovrebbe fregartene qualcosa}
        \textbf{tl;dr} Il codice \LaTeX di questi appunti è pubblico e modificabile all'hyperlink gigante in basso.
        
        \textbf{Versione lunga}: inizialmente ho creato questi appunti solo per me, poi li ho condivisi, poi ho aggiunto un link di donazioni.
        
        Vedendo un po' di gente usarli e farli girare, ho pensato di poter creare qualcosa di più interessante di un pacco di appunti che fra qualche anno sarà datato e un guadagno di qualche spicciolo per me: vorrei creare un punto di raccolta per tutti gli studenti Unisa e non, per trovare, inserire e aggiornare appunti. In questo modo spero che lo sforzo di gruppo fornisca una fonte gratuita, libera, collettiva e soprattutto aggiornata di conoscenza, anche quando i corsi inevitabilmente verranno aggiornati e quando io, fra circa 12 ere geologiche, mi laureerò.
        
        \vspace{3mm}
        
        Oltre alla partecipazione ai singoli documenti, mi farebbe piacere aggiungere collaboratori per dividerci la gestione della repository, e che possano magari ereditarla del tutto quando perché non ho intenzione di accettare pull request per tutta la vita.
    \begin{figure}[h]
        \centering
        \href{https://github.com/shyimon/UnisaComeBabele}
        {\includegraphics[width=0.9\textwidth]{Images/github.png}}
    \end{figure}